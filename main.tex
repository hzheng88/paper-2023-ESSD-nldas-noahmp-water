\documentclass[essd]{copernicus}

\begin{document}

\title{A 48-member perturbed-physics ensemble of 1/8\degree{} terrestrial water budget data over the conterminous United States, 1980--2015}

\Author[1]{Hui}{Zheng}
\Author[2]{Zong-Liang}{Yang}
\Author[1,3]{Wenli}{Fei}
\Author[4,2]{Peirong}{Lin}
\Author[2]{Wen-Ying}{Wu}
\Author[5,2]{Jiangfeng}{Wei}
\Author[6,2]{Lingcheng}{Li}
\Author[7,2]{Long}{Zhao}
\Author[1]{Kai}{Li}

\affil[1]{Key Laboratory of Regional Climate-Environment Research for Temperate East Asia, Institute of Atmospheric Physics, Chinese Academy of Sciences, Beijing, 100029, China}
\affil[2]{Department of Geological Sciences, John A. and Katherine G. Jackson School of Geosciences, The University of Texas at Austin, Austin, Texas, 78705, USA}
\affil[3]{University of Chinese Academy of Sciences, Beijing, 100049, China}
\affil[4]{Institute of Remote Sensing and Geographic Information System, School of Earth and Space Sciences, Peking University, Beijing, 100871, China}
\affil[5]{Collaborative Innovation Center on Forecast and Evaluation of Meteorological Disasters/Key Laboratory of Meteorological Disaster, Ministry of Education/International Joint Research Laboratory on Climate and Environment Change, Nanjing University of Information Science and Technology, Nanjing, 210044, China}
\affil[6]{Pacific Northwest National Laboratory, Richland, Washington, 99354, USA}
\affil[7]{School of Geographical Sciences, Southwest University, Chongqing, 400715, China}

\correspondence{Zong-Liang Yang (\href{mailto:liang@jsg.utexas.edu}{liang@jsg.utexas.edu})}

\runningtitle{NLDAS-NoahMP perturbed-physics ensemble hydrological data}

\runningauthor{Zheng, Yang, Fei, et al.}

\firstpage{1}

\maketitle


\begin{abstract}

    Terrestrial water budget (TWB) data over large domains are of high interest for various hydrological applications. Land surface model (LSM) ensembles are well-suited for a spatiotemporally continuous and physically consistent estimation. As an augmentation of the operational North American Land Data Assimilation System phase 2 (NLDAS\nobreakdash-2) four-LSM ensemble, this study presents a 48-member perturbed-physics ensemble configured from the Noah LSM with multi-physics options (Noah-MP). The 48 Noah-MP physics configurations are selected to give a representative cross-section of commonly used LSMs for parameterizing runoff, atmospheric surface layer turbulence, soil moisture limitation on leaf stomata, and stomatal conductance.

    Simulation of the ensemble reproduced 1980–2015 monthly TWB over the conterminous United States at a 1/8° spatial resolution. Total and the constitutes of evapotranspiration (i.e., canopy evaporation, soil evaporation, and transpiration) and runoff (i.e., the surface and subsurface components) are freely downloadable at \url{https://doi.org/place-holder-nmpens-pretrn}. Terrestrial water storage and its components (i.e., snow water equivalent, four-layer soil water content from the ground down to 2-m depth, and groundwater storage anomaly) are retrievable from \url{https://doi.org/place-holder-nmpens-tws}.

    Inter-comparisons and evaluations of the ensemble reconstructions show a larger ensemble size and would facilitate a better quantification and process-level attribution of the reconstruction uncertainty. The others include: (1) improved performance, (2) updated quantification of reconstruction uncertainty, (3) process-level attribution of the estimation uncertainty.

\end{abstract}


\introduction \label{sec:intro}

Estimates of terrestrial water budget (TWB) over continental domains are of high interest for a broad range of hydrological applications. Publicly available data have been applied to investigate the state of the terrestrial water cycle \citep{trenberth2013GRL, rodell2015JC, scanlon2018P, yin2020HESS, pascolini-campbell2021N}, to understand the interactions among hydrological processes, vegetation, climate, and human activities \citep{trenberth2013JC, lafontaine2015JAWRA, ward2014PNAS, levia2020NG}, to examine the availability and variability of water resources and use \citep{wu2021JH, hejazi2014HESS, scanlon2012P, voss2013WRR, lv2019JGRA, le2011P, rodell2009N}, and to assess the risk of extreme events such as droughts \citep{peters-lidard2021BAMS, prudhomme2014P, dai2013NCC, su2021JH} and floods \citep{emerton2017NC, lin2018JH}. With the expanding applications, the availability of TWB datasets has been rapidly increasing \citep{peters-lidard2018MM, saxe2021HESS, zhang2018HESS}.

Commonly used methods of estimating TWB include remote sensing, in-situ observations, and model simulations \citep{saxe2021HESS, mccabe2017HESS, pan2012JC, gao2010IJRS, trenberth2007JH}. Among these methods, land surface models (LSMs) are apt for continuously producing physically consistent TWB over a large domain and long period. These characteristics are particularly favorable for some circumstances. LSMs could estimate various TWB components simultaneously, whereas, for some components, such as runoff \citep{lin2019WRR, beck2017HESS}, root-zone soil moisture \citep{xia2015JHa, xia2015JH}, and transpiration \citep{lian2018NCC}, direct remote sensing is either unavailable or highly uncertain. In remote or topographically complex regions, LSMs are valuable as in-situ observations are sparse \citep{kim2021TC}. Estimation based on remote sensing and in-situ observations is often impeded by scale mismatch and observation gaps, whereas the issues impair LSM simulations less. Besides, LSM simulations are well complementary to remote sensing and in-situ observations. Combinations of the estimates from different techniques can improve the estimation accuracy \citep{zhang2018HESS, pan2012JC}, whereas comparisons between the model-simulated estimates and observations could reveal the impacts of human activities \citep{zaussinger2019HESS} and underground processes \citep{zheng2020JAMES}.

For delivering consistent and continuous TWB estimates over a large domain, several operational multi-LSM ensemble simulation systems have been set up \citep{xia2019JMR, shi2011SCES, carrera2015JH, rodell2004BAMS}.
The North American Land Data Assimilation System (NLDAS) \citep{xia2012JGRA, xia2012JGRAa, mitchell2004JGRA} stands as one of the pioneering and most successful systems. The NLDAS phase 2 (NLDAS\nobreakdash-2) operates over the conterminous United States (CONUS) from 1979 to near real-time at a spatial resolution of 1/8°.
The system generates a set of multi-source synthesized data of surface meteorology, vegetation, and soils and uses them to drive an ensemble of four different LSMs. The four LSMs, namely Noah version 2.8 \citep{ek2003JGRA, chen2001MWR, chen2001MWRa, chen1997BM}, Variable Infiltration Capacity (VIC) version 4.0.3 \citep{liang1994JGRA}, Mosaic \citep{koster1992JGRA}, and Sacramento Soil Moisture Accounting (SAC) \citep{burnash1973}, are selected to give a good cross-section of diverse LSMs with different physical parameterizations \citep{mitchell2004JGRA}.
The models have varying strengths and weaknesses in process parameterizations and modeling skills \citep{kumar2017WRR}. An ensemble of multiple models could produce an aggregated estimate that outperforms most of the ensemble constitutes \citep{fei2021WRR, beck2017HESS, guo2007QJRMS, ajami2007WRR} as well as quantify the estimation uncertainty resulted from different model formulations \citep{troin2021WRR, cloke2009JoH}.
Evaluations of the NLDAS\nobreakdash-2 four-LSM ensemble estimates have shown satisfactory performance in matching the observed evapotranspiration \citep{zhang2020AFM, xia2012JGRA, kumar2018RS}, runoff \citep{xia2012JGRAa}, soil moisture \citep{xia2015JH, xia2015JHa}.

This study enriches the NLDAS\nobreakdash-2 four-model ensemble with 48 perturbed-physics configurations of the Noah LSM with multi-physics options (Noah-MP) citep{niu2011JGRA, yang2011JGRA}. Noah-MP has more physically realistic representations of the vertical stratification than the NLDAS\nobreakdash-2 models have. A column of the land in Noah-MP consists of a vegetation canopy layer, three snowpack layers, four soil layers, and a groundwater component \citep{niu2011JGRA}. Conceptual (e.g., the five water tanks of SAC) and lumped (e.g., the vegetation-soil combined surface layer of Noah) representations of the stratification of vegetation and soil, as used in the NLDAS\nobreakdash-2 models, are minimized. Noah-MP has a more comprehensive representation of various land surface processes that are evident at different depths. The modeled processes include snow accumulation and ablation, infiltration, percolation, retention, freezing/thawing of snow/soil water, groundwater recharge/discharge, and energy constraints \citep{niu2011JGRA}. The improvements in vertical stratification and process parameterizations are expected to produce better TWB estimates. Previous comparisons between Noah-MP and the four NLDAS\nobreakdash-2 LSMs have shown that Noah-MP is comparable or better in estimating soil moisture \citep{cai2014JGRAa}, runoff \citep{fei2021WRR, cai2014JGRAa}, and evapotranspiration \citep{zhang2020AFM}. The results encouraged computationally expensive runs of Noah-MP as done in this study.

The enrichment also features a single-model perturbed-physics ensemble, which is different from the widely-used multi-model ensembles. The Noah-MP ensemble is constructed by shuffling the available parameterization options of several selected processes. The ensemble size grows exponentially as a multiplication of the available parameterization options of different processes \citep{yang2011JGRA, zhang2016JGRA, gan2019WRR}. A large ensemble should give a broad cross-section of feasible model formulations to account for the model uncertainty in TWB estimation \citep{telteu2021GMD, mitchell2004JGRA}) and is critical for a statistically reliable estimation of the probability of hydrological events such as floods and droughts \citep{troin2021WRR}. The single-model perturbed-physics ensemble also facilitates uncertainty attribution and reduction. The ensemble consists of pairs that are different in the parameterization of one process and the same in the other processes. The impacts of the process’s parameterizations can be pinpointed from the model predictive variations. Variance analysis can be applied to quantify the contribution of the parameterization of the process and compare the relative importance of two processes \citep{clark2011WRR, zheng2019WRR}. The quantification could inform further model development to reduce the model uncertainty. However, there are pitfalls unique to the single-model perturbed-physics ensemble. The ensemble members share a significant portion of the same model formulations \citep{fei2021WRR}). The physics similarity among the ensemble members may create redundant information and lower the efficiency characterized by the ratio of ensemble utility over computational costs. Future development of ensemble aggregation and optimization methods should be developed for the single-model perturbed-physics ensemble. A publically available dataset would promote the studies.

We will show the information necessary for using the dataset in Section~\ref{sec:data}, including the dataset variables, file organizations, and the source data and models used for data generation. In this study, we will focus on the assessment of data performance and uncertainty. The assessment methods are described in Section~\ref{sec:evaluation}, and Section~\ref{sec:result} presents the assessment results. The results could directly inform applications of the data. After Section~\ref{sec:availability} shows the data availability, Section~\ref{sec:conclusion} provides conclusions and discussions.


\section{Dataset description} \label{sec:data}

The dataset contains gridded surface energy and water budget variables over the CONUS. Section~\ref{sec:data:variables} describes the dataset variables and their physical relationships. The 48 Noah-MP physics configurations used to create the dataset are detailed in Section~\ref{sec:data:noahmp}. Sections~\ref{sec:data:nldas} and \ref{sec:data:simulation} brief the atmospheric forcing, static parameters of vegetation and soil, and simulation settings.


\subsection{Dataset variables} \label{sec:data:variables}

Table~\ref{tbl:variables} lists the dataset variables. The variables are available at each 1/8° NLDAS\nobreakdash-2 grid indicated by a land-water mask ($X$). The variables fall into two groups: the surface water and energy budget variables.

\begin{table}[t]
    \caption{The dataset variables.}
    \label{tbl:variables}
    \centering
    \begin{tabular}{lll}
        \tophline
        Symbol       & Units                   & Description                               \\
        \middlehline
        \multicolumn{3}{c}{surface water budget}                                           \\
        $E$          & \unit{kg~m^{-2}~s^{-1}} & total evapotranspiration                  \\
        $E_{can}$    & \unit{kg~m^{-2}~s^{-1}} & evaporation of canopy interception        \\
        $E_{gnd}$    & \unit{kg~m^{-2}~s^{-1}} & direct evapotranspiration from soil       \\
        $E_{tran}$   & \unit{kg~m^{-2}~s^{-1}} & transpiration                             \\
        $R$          & \unit{kg~m^{-2}~s^{-1}} & total runoff                              \\
        $R_{srf}$    & \unit{kg~m^{-2}~s^{-1}} & surface runoff                            \\
        $R_{und}$    & \unit{kg~m^{-2}~s^{-1}} & subsurface runoff                         \\
        $W$          & \unit{kg~m^{-2}}        & terrestrial water storage                 \\
        $W_{can}$    & \unit{kg~m^{-2}}        & canopy interception                       \\
        $W_{snow}$   & \unit{kg~m^{-2}}        & snow water equivalent                     \\
        $W_{gw}$     & \unit{kg~m^{-2}}        & unconfined groundwater storage            \\
        $w_{soil,i}$ & \unit{m^3~m^{-3}}       & volumetric soil water content             \\
        [1mm]
        \multicolumn{3}{c}{surface energy budget}                                          \\
        $S$          & \unit{W~m^{-2}}         & absorbed solar radiation                  \\
        $S_{can}$    & \unit{W~m^{-2}}         & canopy absorbed solar radiation           \\
        $S_{gnd}$    & \unit{W~m^{-2}}         & ground absorbed solar radiation           \\
        $I$          & \unit{W~m^{-2}}         & net longwave radiation                    \\
        $I_{can}$    & \unit{W~m^{-2}}         & net longwave radiation at the canopy      \\
        $I_{gnd}$    & \unit{W~m^{-2}}         & net longwave radiation at the ground      \\
        $H$          & \unit{W~m^{-2}}         & sensible heat flux                        \\
        $H_{can}$    & \unit{W~m^{-2}}         & sensible heat flux from vegetation canopy \\
        $H_{gnd}$    & \unit{W~m^{-2}}         & sensible heat flux from the ground        \\
        $T_{snow,i}$ & \unit{K}                & snow layer temperature                    \\
        $T_{soil,i}$ & \unit{K}                & soil layer temperature                    \\
        [1mm]
        \multicolumn{3}{c}{auxiliary variables}                                            \\
        $X$          & \unit{-}                & land-water mask                           \\
        \bottomhline
    \end{tabular}
\end{table}


\subsubsection{surface water budget}

Neglecting horizontal water exchange between adjacent grid, the water budget closure could be obtained among the precipitation ($P$, \unit{kg~m^{-2}~s^{-1}}), evapotranspiration ($E$), runoff ($R$), and terrestrial water storage ($W$) change \citep{zheng2020JAMES}:
\begin{equation}
    P = E + R + \Delta W \text{,}
\end{equation}
where, precipitation ($P$) is from NLDAS\nobreakdash-2 (described in Sections~\ref{sec:data:nldas}) and used as model input in this study.

Noah-MP resolves the constitutes of the water budget terms. Evapotranspiration ($E$) consists of canopy evaporation ($E_{can}$), ground evaoration ($E_{gnd}$), and transpiration ($E_{tran}$):
\begin{equation}
    E = E_{can} + E_{gnd} + E_{tran} \text{.}
\end{equation}
Runoff {$R$} is the sum of a surface and subsurface component:
\begin{equation}
    R = R_{srf} + R_{und} \text{.}
\end{equation}
Terrestrial water storage ($W$) constitutes canopy interception ($W_{can}$), snow water equivalent ($W_{snow}$), groundwater storage in unconfined aquifers ($W_{gw}$), and soil water content in the four model layers ($W_{soil,i}$, \unit{kg~m^{-2}}):
\begin{equation}
    W = W_{can} + W_{snow} + W_{gw} + \sum_{i=1}^4 W_{soil,i} \text{.}
\end{equation}
Soil water storage ($W_{soil,i}$) is not included in the dataset but can be calculated from volumetric soil water content ($w_{i}$) as follows:
\begin{equation}
    W_{soil,i} = \rho_{wat} w_i \Delta z_i
    \quad \text{for} \; i = 1, \cdots, 4
    \text{,} \label{eq:soil-vmc}
\end{equation}
where $\rho_{wat}=1000$~\unit{kg~m^{-3}} is the water density; $\Delta z_1=0.1$~\unit{m}, $\Delta z_2=0.3$~\unit{m}, $\Delta z_3=0.6$~\unit{m}, and $\Delta z_2=1.0$~\unit{m} are the thickness of the four soil layers.


\subsubsection{surface energy budget}

Assuming that there is no horizontal energy exchange, the surface energy budget of a grid cell is closed as follows:
\begin{equation}
    R_n = S + I = H + L + \Delta U \text{,} \label{eq:energy-balance}
\end{equation}
where $R_n$ is net radiation, $S$ is absorbed solar radiation, $I$ is net longwave radiation, $H$ is sensible heat flux, $L$ is latent heat flux, and $\Delta U$ (\unit{J~m^{-2}~s^{-1}}) is internal energy change of the land column.
The reflected solar ($S_u$, \unit{W~m^{-2}}) and emitted longwave ($I_u$, \unit{W~m^{-2}}) is not included in the dataset but can be calculated using the NLDAS\nobreakdash-2 downward solar ($S_u$, \unit{W~m^{-2}}) and longwave ($I_u$, \unit{W~m^{-2}}) radiation data:
\begin{align}
    S_u = & S_d - S \text{,} \\
    I_u = & I_d - I \text{.}
\end{align}

The energy budget terms aggregate two components.
\begin{align}
    S = & S_{can} + S_{gnd} \text{,} \\
    I = & I_{can} + I_{gnd} \text{,} \\
    H = & H_{can} + H_{gnd} \text{,} \\
    L = & L_{can} + L_{gnd} \text{,}
\end{align}
where $S_{can}$ ($S_{gnd}$) is the solar radiation absorbed by the canopy (ground), $I_{can}$ ($I_{gnd}$) is the net longwave radiation at the canopy (ground), $H_{can}$ ($H_{gnd}$) is the sensible heat flux from the canopy (ground). $L_{can}$ is the latent heat flux from the canopy and linked to the processes of canopy evaporation ($E_{can}$) and transpiration ($E_{tran}$)). $L_{gnd}$ is the latent heat flux from the ground and linked to ground evaporation ($E_{gnd}$). The canopy and ground components of the energy budget terms are closed as follows:
\begin{align}
    S_{can} + I_{can} = & H_{can} + L_{can} \text{,}            \\
    S_{gnd} + I_{gnd} = & H_{gnd} + L_{gnd} + \Delta U \text{.}
\end{align}

The change in internal energy ($\Delta U$) is associated with the temperature change of and water freezing/melting in the land column:
\begin{equation}
    \begin{split}
        \Delta U = & \sum_{i=1}^{N_{snow}} c_{snow,i} \Delta T_{snow,i} + \sum_{i=1}^{4} c_{soil,i} \Delta T_{soil,i} \\
        & + \Delta U_{phase} \text{,}
    \end{split}
\end{equation}
where $0<=N_{snow}<=3$ is the number of snow layers, $T_{snow,i}$ is the temperature of the $i$-th snow layer, $T_{soil,i}$ is the temperature of the $i$-th soil layer, $\Delta U_{phase}$ is the latent heat associated with water freezing/melting.
$c_{snow,i}$ (\unit{J~K^{-1}~m^{-1}}) is the heat capacity of the $i$-th snow layer and calculated using the specific heat capacity of snow ice ($C_{ice} = 2.108 \times 10^3$ \unit{J~kg^{-1}~K^{-1}}) and supercoolding water ($C_{wat} = 4.188 \times 10^3$ \unit{J~kg^{-1}~K^{-1}}):
\begin{equation}
    c_{snow,i} = ( C_{ice} \rho_{ice} w_{ice,i} + C_{wat} \rho_{wat} w_{liq,i} ) \Delta z_{snow,i} \text{,}
\end{equation}
where $\Delta z_{snow,i}$ (\unit{m}) is the thickness of $i$th snow layer, $w_{ice,i}$ (\unit{m^3~m^{-3}}) is the volumetric snow ice content, $w_{liq,i}$ (\unit{m^3~m^{-3}}) is the volumetric liquid water content, $\rho_{ice} = 916.2$ \unit{kg~m^{-3}} is the ice density.
$c_{soil,i}$ (\unit{J~K^{-1}~m^{-1}}) is the heat capacity of the $i$th soil layer, which is a mixture of dry matter, water, ice, and air.
\begin{equation}
    \begin{split}
        c_{soil,i} = & C_{mine} (1 - w_{max,i}) \\
        & + C_{ice} \rho_{ice} w_{ice,i} \Delta z_{soil,i}
        + C_{wat} \rho_{wat} w_{liq,i} \Delta z_{soil,i} \\
        & + C_{p,air} \rho_{air} (w_{max,i} - w_{ice,i} - w_{liq,i}) \Delta z_{soil,i} \text{,}
    \end{split}
\end{equation}
where $w_{max,i}$ (\unit{m^3~m^{-3}}) is the soil porosity, $C_{p,air} = 1004.64~\unit{J~kg^{-1}~K^{-1}}$ is the isobaric heat capacity of dry air, $\rho_{air}$ is the air density. $C_{mine}$ (\unit{J~m^{-3}~K^{-1}}) is the volumetric heat capacity of soil, which is set as $2\times10^6$~\unit{J~m^{-3}~K^{-1}} in Noah-MP. Note that the volumetric water content ($w_{i}$) shown in equation~(\ref{eq:soil-vmc}) is the sum of volumetric frozen ($w_{ice,i}$) and liquid ($w_{liq,i}$) water content (i.e., $w_{i} = w_{ice,i} + w_{liq,i}$).


\subsection{48 Noah-MP physics configurations} \label{sec:data:noahmp}

\subsubsection{Runoff}

\begin{align}
    R_{srf} = & Q_{in} [(1 - f_{frz}) f_{sat} + f_{frz} ] \text{,}                                  \\
    f_{sat} = & f_{sat,max} \exp(-0.5 f (z_{wt} - z_{bot})) \text{,}                                \\
    f_{frz} = & \exp\left(-\alpha(1 - \frac{w_{ice,1}}{w_{max,1}})\right) - \exp(-\alpha)) \text{,} \\
    R_{sub} = & R_{sub,max} \exp\left( -\Lambda - f(z_{wt} - z_{bot}) \right) \text{,}
\end{align}

\subsubsection{Stomatal conductance}

\subsubsection{Soil moisture limitation on photosynthesis}

\begin{equation}
    \beta = \sum_{i=1}^{N_{root}} \frac{\Delta z_{soil,i}}{z_{root}} \min\left(1, \frac{w_{liq,i} - w_{wilt}}{w_{ref} - w_{wilt}}\right) \text{,}
\end{equation}

\begin{equation}
    \beta = \sum_{i=1}^{N_{root}} \frac{\Delta z_{soil,i}}{z_{root}} \min\left(1, \frac{\psi_{wilt} - \psi_{i}}{\psi_{wilt} - \psi_{sat}}\right) \text{,}
\end{equation}

\begin{equation}
    \beta = \sum_{i=1}^{N_{root}} \frac{\Delta z_{soil,i}}{z_{root}} \min\left(1, 1 - e^{-c^2 \ln\left(\frac{\psi_{wilt}}{\psi_{i}} \right)} \right) \text{,}
\end{equation}

\subsubsection{Near-surface turbulence}

\begin{align}
    \begin{split}
        C_h = & \kappa^2 \left[ \ln\left(\frac{z}{z_{0m}}\right) - \Psi_{m}\left(\frac{z}{L}\right) + \Psi_{m}\left(\frac{z_{0m}}{L}\right) \right]^{-1} \\
        & \left[ \ln\left(\frac{z}{z_{0h}}\right) - \Psi_{h}\left(\frac{z}{L}\right) + \Psi_{h}\left(\frac{z_{0h}}{L}\right) \right]^{-1} \text{,}
    \end{split}                                      \\
    z_{0h} = & z_{0m} \exp\left(-\kappa C \sqrt{\text{Re}^*}\right)
\end{align}

\begin{align}
    \begin{split}
        C_h = & \kappa^2 \left[ \ln\left(\frac{z-d_0}{z_{0m}}\right) - \Psi_{m}\left(\frac{z-d_0}{L}\right) \right]^{-1} \\
        & \left[ \ln\left(\frac{z-d_0}{z_{0h}}\right) - \Psi_{h}\left(\frac{z-d_)}{L}\right) \right]^{-1} \text{,}
    \end{split} \\
    z_{0h} = & z_{0m} = 0.65 z_{ct}
\end{align}

\subsection{NLDAS-2 atmospheric forcing and static parameters} \label{sec:data:nldas}

Static parameters

Atmospheric forcings

The Noah-MP LSM version 3.6 is used.

Details of the NLDAS multi-model ensemble can be found in \citet{xia2012JGRA,xia2012JGRAa,fei2021WRR}.

Details of the Noah-MP multi-parameterization ensemble can be found in \citet{zheng2019WRR,zheng2020JAMES,fei2021WRR}.


\subsection{Simulation settings} \label{sec:data:simulation}


\section{Evaluation and intercomparison methods} \label{sec:evaluation}


\section{Results} \label{sec:result}

\subsection{Spread}

\subsection{Performance}
Biases have been evaluated by \citet{zheng2020JAMES}.

\subsubsection{Terrestrial water storage}

\citep{landerer2012WRR}


\subsubsection{Evapotranspiration}

\subsubsection{Runoff}


\codedataavailability{ The data are freely available for download from the Zenodo online repository at \url{https://doi.org/place-holder-nmpens-pretrn} and \url{https://doi.org/place-holder-nmpens-tws}. The dataset (along with datasets on which it is based) is subject to a Creative Commons BY (attribution) license agreement (\url{https://creativecommons.org/licenses}, last access: 2021-08-16). } \label{sec:availability}


\conclusions[Conclusions and Discussion] \label{sec:conclusion}
conclusion text


\authorcontribution{ZLY initiated and funded the study. HZ conducted the simulation and generated the data. WF contributed to the generation of the figures. WYW, PL, and JW contributed to the validation of the data. All authors contributed to the drafting of the manuscript.}


\competinginterests{The authors declare that they have no conflict of interest}

\disclaimer{The data are provided as-is with no warranties.}

\begin{acknowledgements}

    This work is supported by the National Science Foundation of China (grants 42075165, 41375088, and 41605062).

\end{acknowledgements}


%% REFERENCES
\bibliographystyle{copernicus}
\bibliography{library.bib}

\end{document}
